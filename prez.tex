% arara: xelatex: { shell: yes }
\documentclass{beamer}

\usepackage[czech]{babel}
\usepackage[utf8]{inputenc}
\usepackage{pgfplots}
\usepackage{svg}
\usepackage{minted}

\newminted{yaml}{
	fontsize=\footnotesize, 
	gobble=4,
	frame=lines,
	bgcolor=bg,
	framesep=3mm
} 
\usemintedstyle{yaml}
               		

\setsvg{svgpath = img/}

\usepgfplotslibrary{dateplot}

\usetheme{metropolis}
\title{Systém kontinuální integrace (CI system)}
\date{\today}
\author{Autor: Martin Franc \\ Vedoucí práce: Ing. Jakub Jirůtka \\}
% \institute{FIT ČVUT}

\metroset{block=fill}

\begin{document}


\definecolor{bg}{rgb}{0.95,0.95,0.95}

\defverbatim[colored]\exampleCode{
\begin{yamlcode}
    before_script:
        - apt-get update && apt-get install php composer
        - php -v
        - composer install

    check-syntax:
        script:
            - parallel-lint .

    check-coding-style:
        script:
            - phpcs --standard=ruleset.xml
\end{yamlcode}
}


	\maketitle

	\begin{frame}{Úvod}
		\begin{block}{Kontinuální integrace}
			Technika vývoje softwaru při které jednotliví vývojáři často integrují kód s ostatními, obvykle alespoň jednou za den.
			Každá taková integrace kódu je kontrolována systémem pomocí automatizovaných testů, které detekují možné chyby.
			% http://martinfowler.com/articles/continuousIntegration.html
		\end{block}
		
		Pod automatizovanými testy si lze představit:
		\begin{itemize}
			\item Test syntaktické správnosti
			\item Coding style test
			\item Integrační a unit testy
		\end{itemize}
	\end{frame}

	\begin{frame}{Systém kontinuální integrace}
		Systém kontinuální integrace si lze definovat jako množinu softwaru, která dokáže realizovat následující úkony:

		\begin{enumerate}
			\item Naslouchej změnám v repozitáři verzovacího systému
			\item Stáhni zdrojové soubory projektu
			\item Připrav běhové prostředí na základě konfigurace projektu
			\item Sestav stáhnutý projekt
			\item Spusť testy podle zadání projektu
			\item Informuj výsledku testovaní
		\end{enumerate}
	\end{frame}

	\begin{frame}{CI -- workflow}
		\begin{figure}
			\vbox to \vsize {
				\vfill
				\includesvg[width=\textwidth]{sequence-diagram}
				\vfill
			}
		\end{figure}
	\end{frame}

	\begin{frame}{CI -- běhová prostředí}
		\bigskip\bigskip
		
		\begin{columns}[T,onlytextwidth]
			\column{0.8\textwidth}
				Shell:
				\begin{itemize}
					\item Problémy s bezpečností a úklidem
				\end{itemize}
				Virtuální stroje:
				\begin{itemize}
					\item Kompletní odstínění od systému
					\item Nároky na systémové prostředky
				\end{itemize}
				Kontejnery:
				\begin{itemize}
					\item Nutná podpora ze strany operačního systému
					\item Rychlost a alokace pouze potřebných sytémvých prostředků
				\end{itemize}
			\column{0.2\textwidth}
				\vbox to 40pt{}
				\includesvg[width = 50pt]{virtualbox-logo}
				\includesvg[width = 80pt]{docker-logo}
		\end{columns}
		
	\end{frame}

	\begin{frame}{CI -- ukázková konfigurace projektu (YAML)}
		\exampleCode
	\end{frame}

	\begin{frame}{Cíle práce}
		Rešeršní:
		\begin{itemize}
			\item Úvod do techniky kontinuální integrace
			\item Porovnání existujících řešení
		\end{itemize}
		
		Vytvoření minimalistického CI systému splňující následující body:
		\begin{itemize}
			\item Modulární podpora pro různá běhová prostředí
			\item Konfigurace projektu pomocí textového souboru v repozitáři
			\item Možnost paralelizace úlohy v rámci jedné úlohy
			\item Správa projektů přes terminálové rozhraní (SSH)
			\item Sledování stavu aktuálních i minulých úloh pomocí webového a terminálového rozhraní
		\end{itemize}

	\end{frame}

	\begin{frame}{Existující řešení}
		\begin{columns}[T,onlytextwidth]
			\column{0.8\textwidth}
				Travis:
				\begin{itemize}
					\item Hostovaná služba
					\item Konfugurace přímo v projektu
				\end{itemize}
				Jenkins:
				\begin{itemize}
					\item Nutný vlastní server
					\item Vytváření tzv. \uv{Jobs}
					\item Velké množství pluginů
				\end{itemize}
				Buildbot:
				\begin{itemize}
					\item Framework umožňující škálovatelnost
				\end{itemize}
			\column{0.2\textwidth}
				\includesvg[width = 50pt]{travis-logo}
				\bigskip

				\includesvg[width = 50pt]{jenkins-logo}
		\end{columns}
	\end{frame}

	\begin{frame}{Zvolené řešení}
		\begin{itemize}
			\item Implementace v jazyce Python 3.5 a vyšší
			\item Konfigurace projektu pomocí YAML soubory uvnitř projektu
			\item Webové a API (REST) rozhraní pomocí frameworku Flask
			\item Sledování stavu ůlohy v realném čase pomocí WebSocket spojení
			\item Hlavní běhové prostředí nativní Linuxové kontejnery LXC
			\item Vývoj pokrytý automatizovanými testy (CI?)
		\end{itemize}
	\end{frame}

	\begin{frame}[standout]
		Motivace?	
	\end{frame}

	\begin{frame}{Zdroje}
		[1] DUVALL, Paul M., Steve MATYAS a Andrew GLOVER. Continuous integration: improving software quality and reducing risk. Upper Saddle River, NJ: Addison-Wesley, c2007. ISBN 0321336380.
	
		[2] Koutný, Jiří. Vy ještě nepoužíváte Continuous Integration? zdroják.cz. [online]. 3. 2. 2015 [cit. 2016-11-29]. Dostupné z: https://www.zdrojak.cz/clanky/vy-jeste-nepouzivate-continuous-integration/
	
		[3] FOWLER, Martin. Continuous Integration. Martin Fowler. [online]. 1.5.2006 [cit. 2016-11-30]. Dostupné z: http://www.martinfowler.com/articles/continuousIntegration.html
	\end{frame}

	\begin{frame}[standout]
		Děkuji za pozornost. \\
		\bigskip\bigskip\bigskip
		Otázky?
	\end{frame}

\end{document}